\documentclass[serif]{beamer}
\usepackage{xeCJK}
\usepackage{ctex}
\usepackage{amsfonts}
\usepackage{amssymb}
\usepackage{bm}
\usepackage{amsthm,amsmath,amssymb}
\usepackage{mathrsfs}
\usepackage{latexsym}
\usepackage{geometry}
\usepackage{fancyhdr}
\usepackage{mathtools}
\usepackage{hyperref}
\usefonttheme{professionalfonts}%把所有的字体间距变正常,和article一样
\usetheme{Berlin}
\usecolortheme{whale}
\useinnertheme[shadow=true]{rounded}
\usefonttheme[onlymath]{serif}

\title{数位DP}
\author{\href{https://www.cnblogs.com/znpdco}{ZnPdCo}}
\date{2024年2月}

\begin{document}

\begin{frame}
  \titlepage
\end{frame}

\begin{frame}
  \begin{itemize}[<+-| alert@+>]
    \item 简单来说,
    \item 就是要求统计区间[L,R]内
    \item 满足条件的数的个数
    \item 或者能够想到打表部分分做法的
  \end{itemize}
\end{frame}

\section{手机号码}

\begin{frame}
  \begin{itemize}
    \item 求是 $11$ 位数,且不含前导的 $0$,$[L,R]$ 区间内所有满足号码中要出现至少 $3$ 个相邻的相同数字,且号码中不能同时出现 $8$ 和 $4$ 的数的个数。
    \item $10^{10}\le L\le R\le 10^{11}$
  \end{itemize}
\end{frame}

\begin{frame}
  \begin{itemize}[<+-| alert@+>]
    \item $s[l,r]=s[0,r]-s[0,l-1]$。
    \item 从高位到低位 dp,$i=1$ 表示最高位。
    \item 设 $f[i][j][k][0/1][0/1][0/1][num]$ 表示到第 $i$ 位,上一位是 $j$,这一位是 $k$,是否有连续 $3$ 位的相同的数,是否有 $8/4$,这个数是 $num$(用于检查越界)。
    \item 初始化……
  \end{itemize}
\end{frame}

\begin{frame}
  \begin{itemize}[<+-| alert@+>]
    \item 发现储存 $num$ 不现实。
    \item $n=5141$ 中如果 $num=51$(也就是 $num$ 等于 $n$ 的前缀)那么下一位只能填 $[0,4]$,如果 $num$ 小于 $n$ 的前缀,那么下一位可以随便填。
    \item 设 $f[i][j][k][0/1][0/1][0/1][f=0/1]$ 中的 $f=1$ 表示前面等于 $n$ 的前缀,$f=0$ 表示小于 $n$ 的前缀。
  \end{itemize}
\end{frame}

\begin{frame}
  \begin{itemize}[<+-| alert@+>]
    \item 枚举第 $i$ 位为 $j$,从高到低第 $i$ 位为 $s_i$。
    \item $f[i][f=0]\gets f[i-1][f=0] \qquad {\rm anywise}$
    \item $f[i][f=0]\gets f[i-1][f=1] \qquad {j<s_i}$
    \item $f[i][f=1]\gets f[i-1][f=1] \qquad {j=s_i}$
  \end{itemize}
\end{frame}

\section{里程计}
\begin{frame}
  \begin{itemize}
    \item 找出 $[l,r]$ 的区间里有多少个数满足:有一个数在每一位上出现的次数超过了位数的一半。
    \item $100\le L\le R\le 10^{18}$
  \end{itemize}
\end{frame}

\begin{frame}
  \begin{itemize}[<+-| alert@+>]
    \item 枚举超过位数一半的数,分开处理,这样我们的问题就变简单了,每次只需要记录一个数的出现次数就完了。
    \item 完了吗??当然没有。比如 $1122$,我们算 $1$ 的时候会算一次,算 $2$ 又要算一次。
    \item 经过总结,发现如果一个数被算了两次,那么这样子的数满足位数是偶数且只由两个数字构成,所以我们在最后再来一次数位 DP,找到满足这个条件的数,减去就完事了。
  \end{itemize}
\end{frame}

\section{淘金}
\begin{frame}
  \begin{itemize}
    \item 有一个 $X$、$Y$ 轴坐标范围为 $1\sim n$ 的范围的方阵,每个点上有块黄金。一阵风来 $(x,y)$ 上的黄金到了 $(g(x),g(y))$,$g(x)$ 为 $x$ 各位上数字的乘积,如果黄金飘出方阵就没了。求在 $k$ 个格子上采集黄金最多可以采集的黄金数。
    \item $1\le n\le 10^{12}$
    \item $k\le \min(n^2,10^5)$
  \end{itemize}
\end{frame}

\begin{frame}
  \begin{itemize}[<+-| alert@+>]
    \item 发现 $g$ 的个数不会太多,大概在 $10^5$ 左右。
    \item 所以用 $f[i][j]$ 表示枚举到第 $i$ 位,数位乘积为 $j$(离散化后)的数的个数。
    \item 比如 $g(0001)$,数位乘积为 $0$,实际上结果为 $1$。
    \item 如何解决?
  \end{itemize}
\end{frame}

\begin{frame}
  \begin{itemize}[<+-| alert@+>]
    \item 一是从低到高位枚举 dp,思路大致一样。
    \item 二是记录当前是否是前导零。
  \end{itemize}
\end{frame}

\begin{frame}
  \begin{itemize}[<+-| alert@+>]
    \item 贪心选择,大根堆……
  \end{itemize}
\end{frame}

\section{数数}
\begin{frame}
  \begin{itemize}
    \item 称一个正整数 $x$ 是幸运数,当且仅当它的十进制表示中不包含数字串集合 $s$ 中任意一个元素作为其子串。例如当 $s = \{22, 333, 0233\}$ 时,$233$ 是幸运数,$2333$、$20233$、$3223$ 不是幸运数。给定 $n$ 和 $s$,计算不大于 $n$ 的幸运数个数。
    \item $n$ 没有前导 $0$,但是 $s_i$ 可能有前导 $0$。
    \item $1 \leq n < 10^{1201}$
    \item $1 \leq m \leq 100$
    \item $1 \leq \sum_{i = 1}^m |s_i| \leq 1500$
  \end{itemize}
\end{frame}

\begin{frame}
  \begin{itemize}[<+-| alert@+>]
    \item 多模式串匹配?
    \item AC 自动机?
  \end{itemize}
\end{frame}

\begin{frame}
  \begin{itemize}[<+-| alert@+>]
    \item 建好自动机后用 $f[i][j]$ 表示第 $i$ 个数位,到第 $j$ 个节点。
    \item 好了吗?当然没有。比如说 $s={012}$,那么 $12$ 本质上是没有问题的,但是实际上我们会枚举为 $012$,匹配了。
    \item 记录一个 $0/1$ 记录当前是否是前导零。如果是则不更新 $j$。
  \end{itemize}
\end{frame}

\section{技巧、题目、练习……}
\begin{frame}
  \begin{itemize}
    \item 求不含前导的 $0$,$[L,R]$ 区间内所有满足号码中要出现至少 $3$ 个相邻的相同数字,且号码中不能同时出现 $8$ 和 $4$ 的数的个数。
    \item $100\le L\le R\le 10^{44}$
  \end{itemize}
\end{frame}

\begin{frame}
  \begin{itemize}[<+-| alert@+>]
    \item 高精度不好写,可以这么写:
    \item $s[l,r]=s[0,r]-s[0,l]+check(l)$。
  \end{itemize}
\end{frame}

\begin{frame}
  \begin{itemize}
    \item 思考:
    \item 给定 $n,m,k$,求 $(\sum_{i=1}^n\sum_{j=1}^n\varphi(i\oplus j\oplus k)+\mu(i\oplus j\oplus k)+d(i\oplus j\oplus k))\bmod 998,244,353$。
    \item $1\le n,m,k\le 10^{10}$
  \end{itemize}
\end{frame}

\end{document}